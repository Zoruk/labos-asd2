\documentclass[a4paper,11pt]{article}

\usepackage[utf8]{inputenc}
\usepackage[T1]{fontenc}
\usepackage{lmodern}
\usepackage{tablefootnote}

\usepackage[frenchb]{babel}

%\setlength{\textwidth}{16cm}
%\setlength{\textheight}{24cm}
%\setlength{\oddsidemargin}{0cm}
%\setlength{\voffset}{-1.5cm}

\title{Compression avec l'algorithme de Hauffman}
\author{Loïc Haas \& Romain Maillard}

\begin{document}
\maketitle
\newpage
\section{Réalisation du code}
La structure des classes étant imposée la réalisation du code était relativement vite faite.
\subsection*{Amélioration possible}
\begin{itemize}
	\item Format du stockage de l'arbre de Hauffman, par exemple le linéariser et stoker uniquement les valeur binaires des octets de chaque feuille.
	\item Pour la décompression utiliser un tableau avec comme index les valeurs valeurs binaire de chaque feuille de l'arbre.
	\item Lors de la décompression lire plus d'information dans le buffer. Par exemple lire kilo octet par kilo octet au-lieu de le faire par octet.
\end{itemize}
\section{Comparaison des résultats}
\subsection*{Résultats obtenus}
\begin{table}[h]
\begin{tabular}{|c|c|c|c|}
\hline 
Format de fichier & Taille initiale & Taille après compression & Proportion de compression\tablefootnote{Proportion par rapport a la taille initiale par exemple si le fichier initiale fait 1024Ko et le fichier de destination fais 512Ko la proportion sera de 50\%} \\
\hline 
txt & 1015k & 576k & 57\% \\ 
txt & 79o & 251o & 318\% \\ 
bmp & 23829ko & 6381ko & 27\% \\
jpg & 302k & 287k & 95\% \\  
pdf & 28163o & 29250o & 103\% \\
pdf & 13386ko & 12305ko & 91\% \\
flac & 21329ko & 21311ko & 100\% \\
\hline 
\end{tabular}
\end{table}
\subsection*{Discutions des résultats}

\end{document}